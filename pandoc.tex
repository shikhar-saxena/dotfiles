% \usepackage{titlesec}
% \usepackage{titling}
\usepackage{tcolorbox}
\usepackage[ruled]{algorithm2e}

% \usepackage{tikz}
% \usetikzlibrary{automata,arrows,positioning,calc}

% \setmainfont{Libertinus Sans}
% \setmathrm{Fira Math}

% Specify different font for section headings
% \newfontfamily\headingfont[]{Gill Sans}
% \titleformat*{\section}{\LARGE\headingfont}
% \titleformat*{\subsection}{\Large\headingfont}
% \titleformat*{\subsubsection}{\large\headingfont}
% \renewcommand{\maketitlehooka}{\headingfont}

\usepackage{sectsty}
\allsectionsfont{\sffamily}

\usepackage{enumitem}
\setlist[itemize,1]{label=$\circ$}
\setlist[itemize,2]{label=$\star$}
\setlist[itemize,3]{label=$\bullet$}

\usepackage{graphicx}

% \renewcommand{\familydefault}{\sfdefault}
% \usepackage{cmbright}
% \renewcommand*\rmdefault{cmss}
\renewcommand*\ttdefault{cmtt}
% \usepackage{isomath}

\usepackage{amsthm, amssymb}
\newtheorem{definition}{Definition}
\newtheorem{prop}{Proposition}
\newtheorem{corollary}{Corollary}
\newtheorem*{note}{Note}
\newtheorem{theorem}{Theorem}
\newtheorem{example}{Example}
\newtheorem{lemma}{Lemma}
\newtheorem*{remark}{Remark}

% \usepackage{parskip}
% \setlength{\parindent}{15pt}


% \usepackage{tocloft}
% %\renewcommand\cftsecfont{\normalfont\bfseries}
% %\renewcommand\cftsubsecfont{\normalfont}
% %\renewcommand\cftsubsubsecfont{\normalfont\bfseries}
% \renewcommand{\cftsecdotsep}{2}
%
% \renewcommand{\cftsecleader}{\color{red}       \cftdotfill{\cftsecdotsep}}
% \renewcommand{\cftsubsecleader}{\color{red}     \cftdotfill{\cftsecdotsep}}
% \renewcommand{\cftsubsubsecleader}{\color{red}     \cftdotfill{\cftsecdotsep}}

% \usepackage{fancyhdr}
% \pagestyle{fancy}
% % \fancyhf{}
% % \fancyhead[L]{\thesection}
% % \fancyhead[R]{\ass}
% % \fancyfoot[C]{\thepage}
% \renewcommand{\footrulewidth}{0.4pt}
% \renewcommand{\headrulewidth}{0.4pt}
%
% \fancypagestyle{plain}{%  the preset of fancyhdr 
%   % \fancyhf{}
% \fancyhead[L]{\thetitle}
% % \fancyhead[R]{\ass}
% % \fancyfoot[C]{\thepage}
% }


\newcommand{\ra}{\rightarrow}
\newcommand{\la}{\leftarrow}
\newcommand{\far}{\infty}
\newcommand{\bs}[1]{\left[#1\right]}
\newcommand{\inn}[2]{\left\langle #1, #2 \right\rangle}
\newcommand{\bp}[1]{\left(#1\right)}
\newcommand{\bc}[1]{\left\{#1\right\}}
\newcommand{\lamb}{\lambda}
\newcommand{\epsi}{\epsilon}


\DeclareMathOperator{\diag}{diag}

\input{/home/shikhar/math_commands.tex}
